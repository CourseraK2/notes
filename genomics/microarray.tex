\documentclass[a4paper]{article}
\usepackage[T1]{fontenc}
\usepackage[utf8]{inputenc}
\usepackage{lmodern}
\usepackage[english]{babel}
\usepackage{amsmath}
\begin{document}

Use of Microarray/NGS:\\
To explain phenotypic variation[Whjy are cancer cells affected by some drugs while others not]


Transcription inside the nucleus gives rise to mRNA which goues outside the nucleus to 
be translated into proteins[Amino acids]

What makes every cell different[Refer notes.tex]: Gene expression is different for 
each cell. 

why shouild a liver cell give rise to only a lover cell? It cannot be coded in the sequence
since this sequence is commeon with the skin cell too. This arises due to epigenetist(x)c

Outliers in data can shift your mean and standard deviation by high values, a more robust estimator is the median and MAD.

MAD(median absolute deviation) = 1.4826*median$(abs($X_i$-median($X$)))
Consider $[$2,2,3,4,14$]$, meadian =3 , absolute deviations from median = $[$1,1,0,1,11$]$ abnd now the MAD is 1, thus unaffected by the 
outlier

Pearson correlation: 

For opulation:


\begin{math}
p = \frac{cov(X,Y)}{\sigma_x \sigma_y} = \frac{E(X-\mu_x)(Y-\mu_y)}{\sigma_x \sigma_y}

\end{math}
\begin{math}
For a sample:\\

p = \frac{\sum_{i=0}^n(X_i-\bar{X})(Y_i-\bar{Y})}{\sqrt{\sum_{i=0}^n(X_i-\bar{X})^2} \sqrt{\sum_{i=0}^n(Y_i-\bar{Y})}}



\end{math}


Spearman correlation:\\
This is same as pearson correlation but for ranked variables. the raw scores $X_i$, $Y_i$ are converted to 
ranks 



When the variables are not normally distributed,  or the relationship between the variables is not linear
it would make ore sense to use the Spearman correlation. It takes no assumption about the distribution and hence
 since ranks are being used instead of raw values, the large/small outliers will have no  effect.
If the rankes of X,Y are similar theere is for sure a correlation but that might not be shown by the Pearson coefficient
 because the relationship might not necessarily be linear.

In short use pearson for normally distributed datasets and Spearman otherwise.

Plotting:

Pie is replaced by bar\\

Bar plot woith erros is replaced by box plot\\

Scale is importane(log transform for example)\\

For comparing variables use a scatter plot\\

2D projections are much more telling\\

Why MA plots? \\

You want to tilst your head to find out the outliers? Why not tilt the graph itself


Why do we plot qq plots? : Is the data normally distributed?


Z scores: \\

\begin{math}
Z_i= \frac{X_i_\bar{X}}{sd_X}
\end{math}


Use of Microarrays:\\

1. Gene Expression \\
2. Genotyping SNP array: Is the person AA, AG or GG?: Have probels for allele1, allele2, see which hybridises\\

3. Chip Mircoarray: Given the DNA find out where exactly is a specific protein bound: Fragment the DNA[some pieces have the protein others dont]

Now with the protein bound DNA hyvbridise them to the tiling array. If the probe "lightsup" then that probe has the associate protein 
bound to it


Next Gneration Sequencing:\\

Prepare a sample. Thendadd adapters. These adapters lets your sequences attach themselves to a piece
of solid. Once attached to solid we now amplify each one so that we have millinons of 
copies of each. Now start adding nucleotides which have already been blabelled. This
is not hybridisation but nuclotide elongation. Once the first nucleotide
attaches we take a picture . Now move to next base and keep taking pictures.
Each time you take a picture you read all 4 intensities but the one emitting the max intensity is the
one you just sequenced


apllications: \\


\begin{enumerate}
\item Resequencing
\item SNP discovery and genotyping
\item Variant discovery and quantification
\item TF binding: Chip-Seq
\item Gene expression: RNA-seq
\item Methylation


\end{enumerate}


Variant Detection: Multiple reads with different base at one position \\
RNA-Seq: RNa convert to DNA and then map to reference Gene. This is done for two sample. if you see more reads in sample1 then it is differentially expressed.\\
Ci[Seq: Similar to DNa seq . You take taht part of dNa bound to protein./ Using peak detection find locations that have enough reads to be believed t
bound to protein.



\end{document}
