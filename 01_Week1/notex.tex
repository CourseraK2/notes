\documentclass[a4paper]{article}
\usepackage[T1]{fontenc}
\usepackage[utf8]{inputenc}
\usepackage{lmodern}
\usepackage[english]{babel}
\begin{document}
\begin{itemize}
\item \bf{Anecdotal Evidence} : Evidence based froma a limited sample size which might not be
representative of the population

\end{itemize}
First question you should ask : What is the poulation and what is the sample?
Research Question $\longrightarrow$ Population $\longrightarrow$ Sample $\longrightarrow$ Generalze to

Type of variable:
\begin{itemize}
\item Numerical: TAkes numerical values (Add, subtract on these)
\begin{itemize}
\item Continuous : E.g height
\item Discrete: E.g. age	
\end{itemize}
\item Categorical : Limited number of distinct categories, can be identified with numbers
but no aritmeitc operation
\begin{itemize}
\item Ordinal: Have ordered levels , Satisfactory, poor, very poor
\item Regular : Morning person or Afternoon person
\end{itemize}
\end{itemize}

When variables show some connection with one anotehr they are called associated or dependent
can be neg or postive . Not associated = independent


\begin{itemize}
\item Observatoional Study:  Collect data without affecting how data arises, "observe"
\begin{itemize}
\item Retrospective: Data collected from past
\item Prospective: Collected throught study
\end{itemize}
\item Experimental Study: Randomly assign subjects to treatment, thus can establish causal
connections between the explanatory and observed variables


\end{itemize}

In an observationsal study it is difficult to conclude, unless you really control the effect of other variables. In experimental study dues to rando assignment this is taken care of.


Confoundry variables: Variables that affect both explanatory and response variables 

Why is 'census' not a good idea?:
\begin{itemize}
\item Part of the 'census' might not be representative of the population, e.g. Immigrants
\item Population is dynamic! You dont taste whole of soup to find out its taste! Expolratory analysis!

\item For inference to be valid, the sample should represent populaton. Stir the soup before tasting!

\end{itemize}
Bias:
\begin{itemize}
\item Convenient Sample: Pick up people from your class for study because they are easily accesible
\item Non response: Non-random section of pople respond to your survey. Emailing a survey to 
people who do not have internet connection $=>$ No point!
\item Voluntary Response: Only people who volunteer to respond, respond. not everyone
\end{itemize}
The Literary Digest shut down because its sample was not representative of the population, however large the sample was!
Sampling :\\
\begin{itemize}
\item Simple Random Smapling: Randoml;y select samples so that each data point has equal probaility to get sampled
\item Stratified: Divide population into homogenous groups(strata) and sample from these. Eg. if you want males and females to be equally repreented divide them into males and females and then sample
\item Cluster: Divide into non-homogenous cluster, sample the clusters and then sample the data points. For eg.e divide the geograophy into clusters , travel to few clusters only
\end{itemize}

Principles of experiment design:\\
\begin{itemize}
\item Control: Compare treatment of interest to group
\item Randomize: random assignment of subject to treatment
\item REplicate; replicate entire study or more samples 
\item block: Block vairbales that might affect response variables. Divide pro and amateur atheletes into two groups, then assign them independently to treatment and control and then observe.
\end{itemize}
Blcoking vs explanatory
\begin{itemize}
\item Explantory varrables/factors are conditions we can impose on the experimental units
\item Blocking variables are characteristics that experimental units 	come with that we would like to control
\end{itemize}
\begin{itemize}
\item placebo: fake treamtent
\item placebo effect:eoperiments show effect just because they think they are undergoing treatment
\item blinding: subjects dont know whether they are in control or sample
\item double blind: nor the experimentalist nor subejct knows the groups
\end{itemize}


RAndom sampling" Randomy select subjects from sample
Random assignment: randomly assign subjects to treatement and cotnrol, thus removing the chance of difference

First you sample randomly from the population and then randomy assing them to control and treamtnet groups

RS,RA $=>$ Causal  generalisable [IDEAL experiment]
NO RS, RA $=>$ Causal not generalizable [Most Experiments]
RS, NO RA $=>$ Not causal but generalizable [MOST Observations]
no Rs, no ra $=>$ neither causal nor gneralisable, ONLY Correlational [BAD Observationa]

Scatter plot $=>$ Explanatory varibale on X axis, Response on Y. Association might be identified, causalaity cannot.

Identifying relationship
\begin{itemize}
\item Direction: positve, negative
\item Shape: Linear, curve
\item Strenght: Strong, weak
\item Outliers
\end{itemize}

Skewness is determined by the longer tail!
Modality: Peak in the distribution, unimodal, bimodal, uniform, multimidal

Boxplot: Median and IQR . Using boxplot, you can sketch back the historgram
\end{document}

