\documentclass[a4paper]{article}
\usepackage[T1]{fontenc}
\usepackage[utf8]{inputenc}
\usepackage{lmodern}
\usepackage[english]{babel}
\begin{document}
While plotting a bar graph, it is mandatory tostart 
the y-axis from 0 to prevent the user from giving a false impression
of the differences.

Bar graph obviously leads to some loss of information as we group 
data in classes. The point is this is \textbf{not }meant  for calculations
but is just a way to visually represent the data

Histogram = bar graph for continuous data. The recommendation for determining
how many class sizes are recommendations at the end of day and shouldbe taken with
a pinch of salt.

Frequency polygons are an addition over histogram but should NOT be empolyed
for ordinal data for obvious reasopns, since the polygon implies
constant class size which is not the case given the data ir oirdinal

Cumulative frequncy plygons or OGIVES can be used for determining quantiles and
can start either start from highest oto lowest variable or lowest to highest.
\end{document}

