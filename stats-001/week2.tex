\documentclass[a4paper]{article}
\usepackage[T1]{fontenc}
\usepackage[utf8]{inputenc}
\usepackage{lmodern}
\usepackage[english]{babel}
\begin{document}

Rando Process: You know the 'possible outcomes' , but dont know what it will be.

Frequentist Intepretation:\\
Probability of an event is the proportion of time sevent will occur if the experiment is to be done for infinit times\\

Bayesian Interpretation:\\
Subjective belief. For e,g probability tht it would rain today = 0.4. Prior info/sense builtin.
 Law of large numberS: Collect more data and proportion of occurence of a particular event converged to its probaility
 
 Gamblers' fallcaly/ Lw of averages: Randomization if supposed to compensate for what happened in the past . NO! If you see 10 consectuive heads on toss, the probaility of seeing a head on the 11th toss is also 0.5!
 
 Disjoin(mutually exclusive ) events. P(A and B) = 0 if they are mutually exclusive. I cannot fail and pass together.
 
 Non disjoint = P(a or b) = P(a) + P(b) - P(a and b)
 
 Probability distribution tables:
 \begin{itemize}
\item Events listed must be disjoint
\item probability should be between 0 and 1
\item All should sum to 1	


\end{itemize}

Do sume of two disjoint events always add upto 1? NO!, ther can be other vents as well
\\
Sum of complementary events add upto 1. Complementary events are always disjoint and not vice versa.

 
Two events are disjoint if they cannot happen at the same time. Independent events are one where outcome of one proovides no useful info about the other.

So $P(A and B)=0$ is for disjoint 
and $P(A|B) = A(A)$ is for indepenedent events

Bayes theormed: \\
\begin{math}
P(A|B) = \frac{P(A and B)}{P(B)}
\end{math}

\begin{math}
Z = \frac{obseravtion -mean}{SD}
\end{math}
S core of a mean = 0	

Percentile = are under curve till that observation point

For R:
 \begin{itemize}
 \item pnorm: Given score find percentile
 \item qnorm: Given percentle find score
 
  \end{itemize}
 
 
 When an individual trial has only two possible outcomes it is called
 Bernoulli RV
 
 dbinom(k,n,p) in R gives probability of k successes with proability =p in n  trials
 
 For Binomial :
 \begin{itemize}
 \item $\mu=np$
 \item $\sd=\sqroot{np(1-p)}
 \end{itemize}
 
 Success-failure rule: A binomical distribution with nearly 10 successed and 10 failers is nearly normal.
 
 np $>=$10
 n(1-p) $>=$10

\end{document}

