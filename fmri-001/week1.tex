\documentclass[a4paper]{article}
\usepackage[T1]{fontenc}
\usepackage[utf8]{inputenc}
\usepackage{lmodern}
\usepackage[english]{babel}
\begin{document}
Structural Brain imagin: Study of brainstructire and diagnois

Involves:
\begin{itemize}
\item Computed axial tomography(CAT)
\item Magnetic resonance imaging(MRI)
\item Positron Emission tomography(PET)
\end{itemize}

Functional Brain imaging: used for both cognitive and affective processes
\begin{itemize}
\item PET
\item fMRI
\item Electroencephalography(EEG)
\item Magnetoencephalography(MEG)
\end{itemize}

Limited by spacial resolution, time resolution and invasiveness

\bf{Hemodynamic response function(HRF)} represents change in fMRI signal trigerred by nueronal activity

MR scanner has 1.5-7 Tesla of magnetic field! Earths is 0.00005T

The brains's map is created using the magnetic field generated by measuring the longitudianl and trasverse components of the field
A radio freuqncy pulse is used to align these rotating vectors in the same phase. such that
the vecotrs point in the transverse directiom instead of logitudinal

Longitudinal relaxtion: Restoration of net magnetization along the longitudinal direction
as spins return to parallel state. Exponential growth described by T1

Transverse relaxtion: loss of net magnetiztion in the transverse plane due to loss of coherence. exponential decay desribed by T2

How often nucleis is exvited = TR \\
How soon data collected after exciteation = TE\\
\begin{equation}
Measured signal = M_0(1-e^{\frac{-TR}{T_1}})e^{\frac{-TE}{T_2}}
\end{equation}
T1,T2 are tissue prroperties. You control TR,TE.

$T2^{*}$ is the  cmbined feect of T2 and local inhomogenties, sensitive ot flow and oxygenationcan be used to image brain function

BOLD fmRI: remove inhomegenities
\end{document}
