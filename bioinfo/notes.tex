\documentclass[a4paper]{article}
\usepackage[T1]{fontenc}
\usepackage[utf8]{inputenc}
\usepackage{lmodern}
\usepackage[english]{babel}

\begin{document}
\noindent
\textbf{Genotype Frequency}: Ratio of number of individuals of a given phenotype to the total population.

There is inherent difference between allele frequency and genotype frequency:

Let  allele frequency = $f$ \\
Let Genotype frequency be represented by $G$ \\

Consider diploaid specied (some pink flower) with alleles $'Rw'$

30 flowers with RR\\
20 flowers with Rw\\
50 flowers with ww\\

\begin{math}
f(w) = \frac{Rw+2*(ww)}{2*RR+2*ww+2*Rw} [\frac{Number of  alleles}{Total alleles}]
\end{math}

On the other hand, the genotype freuqcy is :\\
\begin{equation}
g(Rw)=\frac{Rw}{Rw+RR+ww}[\frac{Genotype}{Total polpulation}]
\end{equation}

Thus genotpyic frequency is an indicative of richment of population in terms of a particular genotype.In our case
the $g(Rw)$ is $\frac{30}{100}$


\end{document}

